\title{RTM interpolation using time-shift gathers}
\author{Zhiguang Xue$^*$, Sergey Fomel, and Junzhe Sun, The University of Texas at Austin}
\maketitle

\begin{abstract}
We present an approach for increasing imaging resolution of reverse-time migration (RTM) by taking advantage of time-shift gathers.
The method has two steps: migrating seismic data with the extended imaging condition to get time-shift gathers
and accumulating the time-shift gathers along time-shift axis 
after they are transformed to zero-lag time-shift by a post-stack depth migration on a finer grid.
The final image is generated on a grid which is denser than that of the original time-shift images.
The proposed method is based on the observation that non-zero-lag time-shift images recorded by the regular computing grid
contain the information of zero-lag time-shift image of a denser grid, and such information
can be continued to zero-lag time-shift and refocused at the correct locations on the denser grid.
The extra computational amount of the method relative to RTM has the computational cost of zero-offset migration,
which is almost negligible when compared to prestack shot-record RTM.
Numerical tests on synthetic models demonstrate that the method can effectively improve RTM resolution.
It can also improve the efficiency of RTM if the source and receiver wavefield extrapolations are performed on a coarse grid.
\end{abstract}

\section{introduction}
Thanks to its ability of dealing with extreme velocity complexity
and imposing no dip limitations on the images, 
reverse-time migration or RTM \cite[]{baysal83,mcmechan83,whitmore83} 
has become the standard method for depth imaging in complex areas
\cite[]{zhang07,etgen09,zhang11,zhu14b,casasanta15}.
The image of RTM is traditionally constructed by taking the zero-lag crosscorrelation of the extrapolated source and
receiver wavefields, which can provide correct kinematics and is simple to implement \cite[]{claerbout85}.
More imaging information can be obtained by applying extended imaging conditions, such as space-shift or time-shift imaging condition
\cite[]{sava06,sava11}.
Space-shift or time-shift common-image gathers can be transformed to angle-gathers 
for migration velocity analysis \cite[]{sava03,biondi04,fomel11,xu11,tang13}.

Time-shift gathers were first related to depth focusing analysis \cite[]{faye86,mackay92,nemeth96}.
Recently they have been used to remove RTM artifacts. 
\cite{kaelin11} devised a fan filter based on time-shift imaging condition to eliminate low frequency artifacts and explained its advantages to
other noise attenuation methods.
\cite{khalil13} proposed a new way to transform time-shift gathers to zero-lag time-shift and
to attenuate RTM noise by stacking the transformed time-shift gathers.
\cite{xu14} formulated the transformation of time-shift gathers as a second-pass migration
and removed coherent noise in TTI migration by using the original migration velocity along the symmetry axis but with different
anisotropic parameters in the second pass migration.

Inspired by the approach of \cite{ng07} who used time-shift gathers in one-way wave-equation migration for interpolation to improve computational efficiency,
we propose an application of time-shift gathers to improve the imaging resolution of RTM.
Instead of interpolating the wavefield just in the depth direction as was conducted by \cite{ng07}, 
we seek to perform the interpolation in all spatial directions.
We use the correction equations of \cite{xu14} to build a linear relationship bewteen high-resolution RTM image on a finer grid 
and time-shift gathers on the original grid.
In the paper, we first use an impulse response experiment to test the correctness of the proposed method, 
and then carry out a numerical test on BP-2007 synthetic data to further validate the applicability of the proposed method.

\begin{comment}
The RTM method based on directly solving the two-way wave equation has been developed more than thirty years ago \cite[]{baysal83,mcmechan83,whitmore83}. 
Dispite its ability of dealing with large lateral velocity variation and imposing no dip limitations on the images, it hasn't found its way
into the mainstream industry until recently, which was mainly caused by its extremely expensive computational cost when compared to other migration methods \cite[]{dussaud08}.
Today's seismic imaging in complex geology requires both RTM and wide-azimuth, multi-azimuth or full-azimuth converage data, meaning the imaging process becomes
significantly more expensive.

To resolve this problem, researchers have developed many methods, either based on high performance tools or by designing new algorithms.
Parallel computing has been very commonly used in RTM due to its characteristic of easily being parallelized. 
For example, \cite{chu08} used a pseudospectral and finite difference hybrid approach to solve large-scale seismic modeling and RTM problems using MPI
on disributed memory systems.
\cite{foltinek09} gave an overview of an implementation of RTM on heterogeneous multi-core hardware based on GPUs and demonstrated 
the advantage of industrial-scale RTM on GPU hardware.
Phase encoding method \cite[]{morton98,romero00} is also very effective in reducing the computing time for common-shot RTM.
For example, \cite[]{zhang07} introduced a harmonic-source phase-encoding to allow increased efficiency in a delayed-shot RTM.
\cite{perrone09} compared the effects of different shot encoding functions for RTM.
Some algorithms capable of balancing the accuracy and efficiency of RTM are also proposed.
For example, \cite{fomel13} proposed to use lowrank approximation to implement  wave extrapolation, which provides a very flexible control 
between the accuracy and efficiency of RTM by setting different rank values.

However, phase encoded data causes strong crosstalk in RTM images,
and the algorithms capable of compromising accuracy for efficiency have a very limited efficiency gain.
\end{comment}

\section{Theory}
\cite{sava06} formulate the time-shift imaging condition as
\begin{equation}
I(\mathbf{x},\tau) = \int U_s(\mathbf{x},t+\tau) U_r(\mathbf{x},t-\tau) dt \; ,
\label{eq:timeshift}
\end{equation}
where $\mathbf{x}$ is the spatial coordinate vector, $t$ is time, $\tau$ is the time-shift variable, $U_s$ is the source wavefield
and $U_r$ is the receiver wavefield.
RTM with equation~\ref{eq:timeshift} produces time-shift gathers, which have one extra dimension $\tau$ 
comparing to the result of conventional imaging condition.
When $\tau=0$, the image is simply the conventional RTM result. 
At the non-zero-lag, time-shift images exhibit a spatial shift.

To perform RTM interpolation, we want to take advantage of the information along the extra axis $\tau$, 
and transform the time-shift gathers computed on the original grid to an image on a fine grid.
We realize that the images at non-zero-lag time-shifts recorded at the regular grid points
may be shifted from other locations that may correspond to grid points on a fine grid, as illustrated schematically by Figure~\ref{fig:timeshift}.
In Figure~\ref{fig:timeshift}a, three assumed simple time-shift images are shown. 
The image values recorded at each time-shift are indicated by the blue dots.
Because the grid in Figure~\ref{fig:timeshift}a is sparse, the imaging information in each time-shift image is limited.
If we can transform each image from non-zero-lag time-shift to zero-lag 
image as shown in Figure~\ref{fig:timeshift}b, 
the image values will be moved to some new locations. 
Then the zero-lag alignment of all corrected time-shift images will allow us to stack them together to obtain a high resolution image.

\plot{timeshift}{width=0.49\textwidth}{Simple schematic explanation of RTM interpolation.
(a) Images at different time-shifts; (b) time-shift images after transformation to zero-lag and the subsequent stacked image.}

Regarding the image $I(\mathbf{x},\tau)$ at each image point $\mathbf{x}$ as a new time trace, \cite{xu14} proposed to correct $I(\mathbf{x},\tau)$ 
by applying another zero-offset depth migration to them.
The transformation process simply amounts to solving the general acoustic wave equation for $U(\mathbf{x},t)$
with two initial conditions
\begin{equation}
U(\mathbf{x},t=\tau)=I(\mathbf{x},\tau) \; ,
\label{eq:condition1}
\end{equation}
\begin{equation}
\frac{\partial U}{\partial t}(\mathbf{x} , t=\tau)= \frac{\partial I(\mathbf{x},\tau)}{\partial \tau} \; ,
\label{eq:condition2}
\end{equation}
and back propagating to $t=0$.

In our algorithm, the zero-offset migration is implemented on a dense grid, 
which is the grid that will be used to produce the final stacked image.
However, the time-shift gathers are on the original grid.
The transformation from time-shift gathers on the original grid to the input data on fine grid can be done by simple interpolation.
The finer the stacking grid is, the relatively less information the input data provides, 
and the more migration artifacts will be introduced. 
The artifacts can be attenuated by performing the second migration using a least-squares inversion scheme,
analogous to least-squares migration in shot-record imaging
\cite[]{xue14}.
In the following examples, we apply migration of time-shift gathers to get the interpolated images.

The algorithm can be summarized as follows:
\begin{enumerate}
\item RTM of seismic data with time-shift imaging condition to generate time-shift gathers;
\item A second pass migration of time-shift gathers on a finer grid with the velocity used in RTM to transform them to zero-lag time-shift; 
\item Stacking of transformed images to generate the final image on the dense grid.
\end{enumerate}

\section{Examples}
\inputdir{impulse}
\subsection{Impulse Response Test}

We first use an impulse response experiment to test the correctness of the proposed method.
Our toy synthetic model has a size of $6km\times6km$ with a linearly increasing velocity ranging from $2km/s$ to $3.5km/s$.
The input data consists of one trace with multiple Ricker wavelets.
We first compute a RTM image on a dense grid of $dx=dz=10m$,
and use its zoomed-in part as the reference in the following comparisons.
Figure~\ref{fig:imp-cube,imp-cube2} shows the time-shift gathers before and after correction.
The profiles in the front view are the time-shift images when $\tau=-0.1s$.
The image in Figure~\ref{fig:imp-cube} is at a higher location 
and it is shifted to the zero-lag time-shift location after correction.
The time-shift gather at $3km$ shown in the right side of Figure~\ref{fig:imp-cube} is tilted, and the slopes of
its events correspond to the velocity of the model \cite[]{sava06,xu14}, so that the last event has the biggest slope.
All the tilted events become horizontal after correction, which allows for stacking of time-shift images along time-shift axis.

\multiplot{2}{imp-cube,imp-cube2}{width=0.43\textwidth}{Time-shift gathers for an impulse response test
before (a) and after (b) transformation to zero-time-lag.}

We performed two tests. One is computing the time-shift gathers on the grid of $dx=dz=30m$ 
and attempting to get a high-resolution image on the grid of $dx=dz=10m$.
The other one is computing time-shift gathers on an even coarse grid of $dx=dz=40m$ 
and still trying to get the image of the grid of $dx=dz=10m$.
For comparison, we also show the results of a simple post-imaging interpolation. 
Figure~\ref{fig:imp-part3,imp-lpart3,imp-spart3} shows the results of the first test. 
The conventional RTM result has a limited resolution and the grid is so coarse that
the zero-lag cross-correlation of wavefields cannot be recorded consistently along the events (Figure~\ref{fig:imp-part3}). 
Linear interpolation can lead to an image on a finer grid but the interpolated result shows a degradation.
The result of interpolation with time-shift gathers (Figure~\ref{fig:imp-spart3}) is almost the same as the reference one
, which implies that such image interpolation can recover the real imaging information on the finer grid.
The results of the second test are shown in Figure~\ref{fig:imp-part4,imp-lpart4,imp-spart4} and lead us to the same conclusion.
The result of linear interpolation is poor because of aliasing while the proposed method can still get a good result.
This is due to the fact that time-shift imaging interpolation tries to honor the character of wavefield propagation, 
however the post-imaging interpolation does not \cite[]{ng07}.

\multiplot{3}{imp-part3,imp-lpart3,imp-spart3}{width=0.378\textwidth}{Impulse responses for the test of $dx=dz=30m$. (a) Image on the original grid; 
(b) post-imaging interpolation; (c) RTM interpolation with time-shift gathers.}
\multiplot{3}{imp-part4,imp-lpart4,imp-spart4}{width=0.378\textwidth}{Impulse responses for the test of $dx=dz=40m$. (a) Image on the original grid; 
(b) post-imaging interpolation; (c) RTM interpolation with time-shift gathers.}

\subsection{BP TTI}
\inputdir{bptti}

Our second example is from synthetic BP TTI data set \cite[]{shah07}.
We perform lowrank RTM \cite[]{fomel13,sun13} with time-shift imaging condition on the grid of $dx=12.5m$ and $dz=18.75m$, 
and the zero-lag time-shift image is shown in Figure~\ref{fig:bptti-image}.
This time we don't carry out RTM interpolation for the whole imaging area,
but just a small target area, delineated by the blue box in Figure~\ref{fig:bptti-image}.
We attempt to interpolate the original image of the target area to a grid of $dx=dz=6.25m$.
The second migration in this case is TTI zero-offset migration.

Figure~\ref{fig:btg1,btg2,btg3,atg1,atg2,atg3} shows the time-shift gathers at three locations before and after correction.
The titled events in the three gathers become flat after zero-offset migration.
Figure~\ref{fig:bptti-targeto,bptti-target} shows the target on the original grid and on our expected grid density.
We zoom in their left-top and right-bottom corners as indicated by the blue boxes for a better comparison.
Figure~\ref{fig:zone1-portion1,czone1-portion1,zone1-portion2,czone1-portion2} shows the zoomed-in parts.
After the interpolation, the image appears at a higher resolution and the reflectors become more continuous.
From the comparison of the right-bottom corner, we observe that the proposed method is also able to compensate the amplitude.
In comparison with using the grid of $dx=dz=6.25m$ as the original grid, the RTM efficiency for the target
area has been accelerated by a factor of 6.

\plot*{bptti-image}{width=0.80\textwidth}{RTM image on the grid of $dx=12.5m$ and $dz=18.75m$.}
\multiplot*{6}{btg1,btg2,btg3,atg1,atg2,atg3}{width=0.13\textwidth}
{Time-shift gathers at $x=20,\,25,\,30\,km$ before (a-c) and after (d-f) correction.}
\multiplot*{2}{bptti-targeto,bptti-target}{width=0.44\textwidth}
{RTM interpolation. Target on the grid of $dx=12.5m$ and $dz=18.75m$ (a) and on the finer grid of $dx=dz=6.25m$ (b).}
\multiplot*{4}{zone1-portion1,czone1-portion1,zone1-portion2,czone1-portion2}{width=0.40\textwidth}
{Further zoomed-in parts of target area. Left-top corner before (a) and after (b) RTM interpolation; 
right-bottom corner before (c) and after (d) RTM interpolation.}
%\multiplot*{2}{zone1-portion2,czone1-portion2}{width=0.4\textwidth}
%{Zoomed-in part of right-bottom corner of target before (a) and after (b) RTM interpolation;}
%\multiplot{2}{bptti-image,bptti-target}{width=0.9\textwidth}{(a) RTM result on the grid of $dx=12.5m$ and $dz=18.75m$;
%(b) Interpolated image on the grid of $dx=dz=6.25m$ for the target area.}
%\twocolumn
%\multiplot{4}{zone1-portion1,czone1-portion1,zone1-portion2,czone1-portion2}{width=0.45\textwidth}
%{Zoomed-in part of left-top corner before (a) and after (b) RTM interpolation;
%Zoomed-in part of right-bottom corner before (c) and after (d) RTM interpolation;}
%\multiplot{2}{zone1-portion2,czone1-portion2}{width=0.45\textwidth}
%{Zoomed-in part of right-bottom corner before (a) and after (b) RTM interpolation;}
\section{Conclusions}
RTM interpolation using time-shift gathers is straightforward to implement. 
The workflow contains two main steps: RTM with time-shift imaging condition
on the original grid and zero-offset migration of time-shift images on a finer grid.
The proposed method is effective in improving the imaging resolution of the target area, and 
it can also be used to improve the efficiency of RTM if we perform RTM on a coarse grid and then use
the method to interpolate the image to the desired grid.
The exact resolution limits of the proposed approach remain to be investigated.
Applying time-shift imaging condition generates a data cube that has one more dimension.
This means the cube is 4D for 3D RTM, and in this case the storage can be large.
However, this problem is less severe when applying time-shift imaging condition only for the target areas and 
applying traditional imaging condition for other areas.

\section{Acknowledgements}
We thank Yu Zhang and Bing Tang for helpful discussions and 
thank the sponsors of the Texas Consortium for Computational Seismology (TCCS) for financial support.

\onecolumn
\bibliographystyle{seg}
\bibliography{refs}
