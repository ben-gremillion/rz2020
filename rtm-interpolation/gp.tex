\title{Increasing resolution of reverse-time migration using time-shift gathers}
\author{Zhiguang Xue, zhiguangxue@utexas.edu \\
Sergey Fomel, sergey.fomel@beg.utexas.edu \\
Junzhe Sun, junzhesun@utexas.edu}
\maketitle

\address{
Bureau of Economic Geology \\
John A. and Katherine G. Jackson School of Geosciences \\
The University of Texas at Austin \\
University Station, Box X \\
Austin, TX 78713-8924 \\
}

\footnote{Parts of this paper were presented at 2015 SEG Annual Meeting}

\lefthead{Xue, Fomel \& Sun}
\righthead{RTM interpolation}

\begin{abstract}
	Reverse-time migration (RTM) has become an industry standard for imaging in complex geological areas.
	We present an approach for increasing its imaging resolution by employing time-shift gathers.
	The method consists of two steps: 1) migrating seismic data with the extended imaging condition to get time-shift gathers;
	2) accumulating the information from time-shift gathers
	after they are transformed to zero-lag time-shift by a post-stack depth migration on a finer grid.
	The final image is generated on a grid which is denser than that of the original image thus improving the resolution of the migrated images.
	Our method is based on the observation that non-zero-lag time-shift images recorded on the regular computing grid
	contain the information of zero-lag time-shift image on a denser grid, and such information
	can be continued to zero-lag time-shift and refocused at the correct locations on the denser grid.
	The extra computational cost of the proposed method amounts to the computational cost of zero-offset migration,
	and is almost negligible compared to the cost of prestack shot-record RTM.
	Numerical tests on synthetic models demonstrate that the method can effectively improve RTM resolution.
	It can also be regarded as an approach to improve the efficiency of RTM by performing wavefield extrapolation on a coarse grid and generating 
	the final image on the desired fine grid.
\end{abstract}

\section{Introduction}
Thanks to its ability to deal with high velocity complexity
while imposing no dip limitations on the images, 
reverse-time migration or RTM \cite[]{baysal83,mcmechan83,whitmore83gp} 
has become the standard method for depth imaging in complex areas
\cite[]{zhang07gp,etgen09,zhang11,zhu14b,casasanta15gp,zhao16}.
RTM image is traditionally constructed by taking the zero-lag crosscorrelation of the extrapolated source and
receiver wavefields, which can provide correct kinematic information and is simple to implement \cite[]{claerbout85}.
More imaging information can be obtained by applying extended imaging conditions, such as space-shift or time-shift imaging condition \cite[]{sava06,sava11}.
Space-shift or time-shift common-image gathers can be used for migration velocity analysis \cite[]{sava03,biondi04,fomel11}.
Space-shift common-image gathers can be transformed to angle and azimuth gathers, while time-shift gathers can only be transformed to angle gathers, which may be not very useful for 3D migration velocity analysis. 
However, time-shift imaging is more efficient than space-shift imaging, since it only involves a simple time shift prior to the application of the usual imaging cross-correlation.
Disk storage is also reduced, since the output volume depends on only one time-shift parameter instead of several spatial parameters \cite[]{sava06}.

Time-shift gathers were first introduced for depth focusing analysis in one-way migration \cite[]{faye86gp,mackay92,nemeth96gp}.
Recently, they have been utilized to remove RTM artefacts. 
\cite{kaelin11gp} devised a fan filter based on time-shift imaging condition to eliminate low frequency artefacts and explained its advantages to other noise attenuation methods.
\cite{khalil13gp} proposed a new way to transform time-shift gathers to zero-lag time-shift and
to attenuate RTM noise by stacking the transformed time-shift gathers.
\cite{xu14} formulated the transformation of time-shift gathers as a second-pass migration
and removed coherent noise in TTI (Tilted Transverse Isotropy) migration by using the original migration velocity along the symmetry axis but with different
anisotropic parameters in the second pass migration.

Inspired by the approach of \cite{ng07gp} who used time-shift gathers in one-way wave-equation migration for interpolation to improve computational efficiency,
we propose an application of time-shift gathers to improve the imaging resolution of RTM.
Instead of interpolating the wavefield just in the depth direction as proposed by \cite{ng07gp}, 
we seek to perform the interpolation in all spatial directions, which is a more general dimension.
We use the correction equations of \cite{xu14} to migrate the time-shift gathers on the original grid to zero-lag time-shift images on a fine grid, and then stack all the zero-lag time-shift images to generate the final high-resolution RTM image.
In this way, we generate the final RTM image on the finer grid with the wavefield extrapolation on the original grid.

We start by reviewing the theory of time-shift gathers, and propose the theory of RTM interpolation.
Next, we use an impulse response experiment to test the effectiveness of the proposed method, 
and then carry out a numerical test on BP-2007 synthetic data to validate the applicability of the proposed method in complex media.

\section{Theory}

\subsection{Time-Shift Imaging Condition}
\inputdir{sigsbee2a}

\cite{sava06} formulate the time-shift imaging condition as
\begin{equation}
I(\mathbf{x},\tau) = \int U_s(\mathbf{x},t+\tau) U_r(\mathbf{x},t-\tau) dt \; ,
\label{eq:timeshift}
\end{equation}
where $\mathbf{x}$ is the spatial coordinate vector, $t$ is time, $\tau$ is the time-shift variable, $U_s$ is the source wavefield
and $U_r$ is the receiver wavefield.
RTM with equation~\ref{eq:timeshift} produces time-shift gathers, which have one extra dimension $\tau$ 
comparing to the result of the conventional imaging condition. 
When $\tau=0$, the image is simply the conventional RTM result. 

To illustrate the properties of time-shift gathers, we use Sigsbee 2A model \cite[]{paffenholz02}.
Figure~\ref{fig:sig-vmig} shows the migration velocity, and
Figure~\ref{fig:sig-cube} shows time-shift images obtained by applying equation~\ref{eq:timeshift}.
Its front view shows the zero-lag time-shift image, which is the standard RTM image.
The slices along the time-shift axis are the time-shift gathers at different locations.
Figure~\ref{fig:gather1b,gather2b,gather3b,gather4b} shows four time-shift gathers, 
and each contains many tilted events. 
The slope of the events is related to the corresponding migration velocity \cite[]{sava06}.
Figure~\ref{fig:sig-imgn,sig-imgp} plots two time-shift images with $\tau= \, -0.1$ s and $\tau= \, 0.1$ s, respectively.
When the time-shift is negative, the image appears at a higher position than the focusing depth.
When the time-shift is positive, it appears at a lower position.
If we can find a way to transform the non-zero-lag time-shift images to zero-lag, the new time-shift images would map to the true image locations.

\plot{sig-vmig}{width=1.15\textwidth}{Sigsbee 2A velocity model.}
\plot{sig-cube}{width=1.15\textwidth}{Time-shift images (or time-shift gathers).}
\multiplot*{2}{gather1b,gather2b,gather3b,gather4b}{width=0.33\textwidth}
{Time-shift gathers when x=4, 7, 9, 11 km.}
\multiplot{2}{sig-imgn,sig-imgp}{width=1.15\textwidth}{Two time-shift images. (a) Time-shift $\tau$ = -0.1 s and (b) time-shift $\tau$ = 0.1 s.}

\subsection{RTM Interpolation}
\inputdir{.}

To perform RTM interpolation, we want to take advantage of the extra information along the time-shift axis $\tau$. 
We realize that the non-zero-lag time-shift images recorded at the computing grid points
may be shifted from some locations of the zero-lag time-shift image, and these locations may correspond to grid points on a fine grid mesh.
Then we can shift the non-zero-lag time-shift images on the computing (coarse) grid back to the zero-lag time-shift images on a fine grid.
We schematically illustrate this process in Figure~\ref{fig:timeshift}.
In Figure~\ref{fig:timeshift}a, three assumed simple time-shift images are shown. 
The last two time-shift images appear at lower positions because the positive time shifts are assumed.
The image values recorded at each time shift are indicated by the green dots.
Because the grid in Figure~\ref{fig:timeshift}a is sparse, the imaging information in each time-shift image is limited.
If we can transform each image from non-zero-lag time-shift to zero-lag image as shown in Figure~\ref{fig:timeshift}b, 
the recorded image values will be moved to some new locations, which are most likely not in the original grid mesh and can be represented by a new finer grid mesh.
Then, the zero-lag alignment of all corrected time-shift images will allow us to stack them together to obtain a high resolution image.
This logic of improving resolution follows the techniques proposed previously for conventional NMO (Normal Moveout) stacking \cite[]{stark13,regimbal16gp}.

\plot{timeshift}{width=0.9\textwidth}{Simple schematic explanation of RTM interpolation.
(a) Images at different time-shifts; (b) time-shift images after transformation to zero-lag and the subsequent stacked image.}

To transform non-zero-lag time-shift images $I(\mathbf{x},\tau)$ to zero-time-lag, we use the method proposed by \cite{xu14}.
Regarding the time-shift images as input data, \cite{xu14} proposed to correct the time-lag images $I(\mathbf{x},\tau)$ for $\tau\neq 0$ by applying another zero-offset depth migration to them.
The transformation simply amounts to solving the general acoustic wave equation for $U(\mathbf{x},t)$
with two initial conditions
\begin{equation}
U(\mathbf{x},t=\tau)=I(\mathbf{x},\tau) \; ,
\label{eq:condition1}
\end{equation}
\begin{equation}
\frac{\partial U(\mathbf{x} , t=\tau)}{\partial t}= \frac{\partial I(\mathbf{x},\tau)}{\partial \tau} \; ,
\label{eq:condition2}
\end{equation}
and back propagating $U(\mathbf{x},t)$ to $t=0$.
Each time-lag image can be transformed independently and this process is easily parallelized.

In our algorithm, the zero-offset migration is implemented on a dense grid, 
which is the grid that will be used to produce the final stacked image.
However, the time-shift gathers are on the original grid.
The transformation from time-shift gathers on the original grid to the input data on the fine grid can be done by performing simple interpolation or by padding zeros.
The finer the stacking grid is, the relatively less information the input data provides, 
and the more migration artefacts caused by the incomplete data may be introduced. 
Such artefacts can be attenuated by performing the second zero-offset migration using a least-squares inversion scheme,
analogous to least-squares migration in shot-record imaging \cite[]{nemeth99,kuehl03,xue16,hou16}.

The algorithm can be summarized as follows:
\begin{enumerate}
\item RTM of seismic data with time-shift imaging condition to generate time-shift gathers on a coarse grid;
\item A second pass zero-offset migration of time-shift gathers on a finer grid with the velocity used in RTM to transform them to zero-lag time shift; 
\item Stacking of transformed images to generate the final image on the dense grid.
\end{enumerate}

This method can be used to improve the imaging resolution for target reservoir areas.
The extra computational cost of the method is the cost of applying time-shift imaging condition and the cost of zero-offset migration,
both of which are almost negligible when compared to the prestack source and receiver wavefield propagations.
On the other hand, if we perform RTM on the coarse grid and then get the final image on our desired fine grid with the proposed method,
the acceleration rate of the computation in 2D case would approximately be
\begin{equation}
	\label{eq:efficiency}
	N_e \; \approx \; N_x \, N_z \; ,
\end{equation}
where $N_x$ and $N_z$ are the spatial sampling rates.
Based on this computation advantage, the proposed RTM interpolation could also be used to calculate 
the gradients in least-squares RTM or time-domain full-waveform inversion, 
where the iterative calculation of gradients remains expensive.

\section{Examples}
\inputdir{impulse}
\subsection{Impulse Response Test}

We first use an impulse response experiment to test the correctness of the proposed method.
Our toy synthetic model has a size of 6 km $\times$ 6 km with a linearly increasing velocity ranging from 2 km/s to 3.5 km/s, as shown by Figure~\ref{fig:imp-vel}.
The input data consists of one trace with multiple Ricker wavelets.
We compute an initial RTM image on a dense grid of dx = dz = 10 m,
and use its zoomed-in part (Figure~\ref{fig:imp-refpart1}) as the reference in the following comparisons.
Figure~\ref{fig:imp-cube,imp-cube2} shows the time-shift gathers before and after correction.
The profiles in the front view are the time-shift images when $\tau$ = -0.1 s.
The image in Figure~\ref{fig:imp-cube} is at a higher location 
and it is shifted to the zero-lag time-shift location after correction.
The time-shift gather at 3 km shown in the right side of Figure~\ref{fig:imp-cube} is tilted, and the slopes of
its events correspond to the velocity of the model \cite[]{sava06,xu14}, so that the last event has the biggest slope.
All the tilted events become horizontal after correction, which allows for stacking of time-shift images along time-shift axis.

\multiplot{2}{imp-vel,imp-refpart1}{width=0.8\textwidth}
{(a) Linear gradient velocity model and (b) reference image.}
\multiplot{2}{imp-cube,imp-cube2}{width=0.7\textwidth}{Time-shift gathers for an impulse response test
before (a) and after (b) transformation to zero-time-lag.}

We performed two tests. One is computing the time-shift gathers on the grid of dx = dz = 30 m 
and attempting to get a high-resolution image on the grid of dx = dz = 10 m.
The other one is computing time-shift gathers on an even coarser grid of dx = dz = 40 m 
and still trying to get the image of the grid of dx = dz = 10 m.
For comparison, we also show the results of a post-imaging 4th-order spline interpolation. 
Figure~\ref{fig:imp-part3,imp-spline3,imp-spart3} shows the results of the first test. 
The conventional RTM result has a limited resolution and the grid is so coarse that
the zero-lag cross-correlation of wavefields cannot be recorded consistently along the events (Figure~\ref{fig:imp-part3}). 
Post-imaging interpolation can lead to an image on a finer grid but the interpolated result shows a degradation.
The result of interpolation with time-shift gathers (Figure~\ref{fig:imp-spart3}) is almost the same as the reference one, which implies that such image interpolation can successfully recover the real imaging information on the finer grid.
The results of the second test are shown in Figure~\ref{fig:imp-part4,imp-spline4,imp-spart4} and lead us to the same conclusion.
The result of spline interpolation is poor because of aliasing while the proposed method can still get almost the same result as that generated on the dense grid.
This is due to the fact that time-shift imaging interpolation tries to honor the character of wavefield propagation, 
while the post-imaging interpolation does not \cite[]{ng07gp}.

\multiplot{3}{imp-part3,imp-spline3,imp-spart3}{width=0.7\textwidth}{Impulse responses for the test of dx = dz = 30 m. (a) Image on the original grid; 
(b) post-imaging spline interpolation; (c) RTM interpolation with time-shift gathers.}
\multiplot{3}{imp-part4,imp-spline4,imp-spart4}{width=0.7\textwidth}{Impulse responses for the test of dx = dz = 40 m. (a) Image on the original grid; 
(b) post-imaging spline interpolation; (c) RTM interpolation with time-shift gathers.}

\subsection{BP TTI}
\inputdir{bptti50m}

Our second example is based on the synthetic BP TTI data set \cite[]{shah07}, and Figure~\ref{fig:vp} shows the migration Vp model.
We perform low-rank RTM \cite[]{fomel13,sun16} on the grid of dx = dz = 50 m, 
and the zero-lag time-shift image (or conventional RTM image) after removing low-frequency noise with Laplacian filter is shown in Figure~\ref{fig:bptti-image}.
This time we do not carry out RTM interpolation for the whole imaging area,
but just one arbitrarily selected small target area, as delineated by the blue box in Figure~\ref{fig:bptti-image}.
To achieve this goal, we apply time-shift imaging condition for imaging the target area, 
and use conventional cross-correlation imaging condition for other areas.
We attempt to use the time-shift gathers of the target area to interpolate the original image to a grid of dx = dz = 12.5 m.
The second migration in this case is TTI zero-offset migration.

Figure~\ref{fig:btg1,btg2,btg3,atg1,atg2,atg3} shows the time-shift gathers of the target area at three locations before and after correction.
The tilted events in the three gathers become flat after zero-offset migration, even though we can observe some coherent noise and the discontinuity of the flattened events.
For comparison, we also show the result of a post-imaging sinc interpolation.
Figure~\ref{fig:interzone1,scinterzone1} shows the images of the target area on our expected grid density 
obtained by post-imaging interpolation and RTM interpolation, respectively.
By comparison, we can find that post-imaging interpolation cannot resolve 
the obvious low-wavenumber events at the top caused by large spatial sampling,
while RTM interpolation can lead to a high-resolution result.
From the aspect of efficiency, in comparison with using the grid of dx = dz = 12.5 m as the original grid, the RTM efficiency for the target area has been approximately accelerated by a factor of 16.

\plot{vp}{width=0.90\textwidth}{BP 2007 TTI benchmark Vp model.}
\plot{bptti-image}{width=0.90\textwidth}{RTM image on the grid of dx = dz = 50 m. 
The blue box indicates the randomly selected target area.}
\multiplot{6}{btg1,btg2,btg3,atg1,atg2,atg3}{width=0.25\textwidth}
{Time-shift gathers of the target area at x = 20, 25, 30 km before (a-c) and after (d-f) correction.}
\multiplot{2}{interzone1,scinterzone1}{width=0.8\textwidth}
{Interpolation of the target area from dx = dz = 50 m to dx = dz = 12.5 m. 
(a) Post-imaging sinc interpolation; (b) RTM interpolation with time-shift gathers.}

\section{Discussion}
We propose to use time-shift common-image gathers to improve the resolution of RTM.
It is also possible to use space-shift common-image gathers for increasing the resolution, which may require a completely different formulation.
An analytical estimation of the necessary time shift range for the proposed method remains to be investigated.
Time-shift gathers with only short time shifts may not bring enough information to increase the final resolution of the image on the dense grid.
On the other hand, a large time shift range would require larger storage.
In our two numerical examples, the time shift ranges have been chosen to be from -0.2 to 0.2 s.
The exact resolution limits of the proposed approach and its sensitivity to coherent noise (such as multiples) also need to be investigated.

The proposed method assumes that the migration velocity is accurate enough so that the tilted events in time-shift gathers can become flat after the transformation with the method proposed by \cite{xu14}.
When the migration velocity is not correct, it is possible to combine the proposed method with velocity model building methods that are based on time-shift gathers \cite[]{yang11, luo16}.

Although our examples are in 2D, the 3D extension of the proposed method is straightforward, and the improvement of computational efficiency will be even more obvious.
The time-shift image cube of 3D RTM is 4D, and in this case the storage may be an issue.
However, this problem is less severe when applying time-shift imaging condition in a target-oriented fashion.

\section{Conclusions}
The proposed RTM interpolation using time-shift gathers is straightforward to implement. 
The workflow consists of two main steps: RTM with time-shift imaging condition
on the original grid and zero-offset migration of time-shift images on a finer grid.
The proposed method is effective in improving the imaging resolution of the target area, and 
it can also be used to improve the efficiency of RTM if we perform RTM on a coarse grid and then use
the method to interpolate the image to the desired grid.
The reconstruction of seismic events in the impulse response experiment verifies the effectiveness of the proposed approach, and the image improvements in the BP 2007 example confirm the applicability of our method in complex media.
An analytical estimation of the necessary time shift range and the exact resolution limits of the proposed method remain to be investigated.

\section{Acknowledgments}
We thank associate editor G. Lambar\'{e} and two anonymous reviewers for providing valuable comments and suggestions.
We thank the sponsors of the Texas Consortium for Computational Seismology (TCCS) for financial support.
The first and third authors were additionally supported by the Statoil Fellows Program at The University of Texas at Austin.
The BP 2007 benchmark data set was created by H. Shah and released by BP Exploration Operation Co., Ltd.
We thank Texas Advanced Computing Center for providing computational resources used in this study.
All computations presented in this paper are reproducible using Madagascar software package \cite[]{fomel13b}.

\bibliographystyle{seg}
\bibliography{refs}
